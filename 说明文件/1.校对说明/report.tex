%!TEX program = xelatex
\documentclass[10pt,a4paper,UTF8]{article}
\usepackage{ctex}
\usepackage{indentfirst,latexsym,bm,amsmath,amsthm,enumerate,graphicx,setspace,subfigure,cite,fancyhdr,float}
\usepackage[vmargin=2.54cm,hmargin=3.18cm]{geometry}

\begin{document}
%~~~~~~~~~~~~~~~~~~~~~~~~~~~~~~~修正部分~~~~~~~~~~~~~~~~~~~~~~~~~~~~~~~~~~~~~~~~~~~~~~~~~~
    %通用修正
    \setlength{\parindent}{2em}                        %首行缩进2字符
    \renewcommand{\baselinestretch}{1.5}\normalsize    %修改行距为标准的x倍

    %中文相关的修正
    \renewcommand*{\qedsymbol}{[证毕]}                 %中文的证x明毕
    \newcommand{\HEI}{\CJKfamily{hei}}                 %中文黑体
    \newcommand{\KAI}{\CJKfamily{kai}}                 %中文楷体

    %英文修正
    \newcommand{\romannum}[1]{\expandafter{\romannumeral#1}}    %罗马字符

    %添加新的定理环境
    \newtheorem{instance}{\HEI例}[section]

    %~~~~~~~~~~~~~~~~~~~~~~~~~~~~~~~修正结束~~~~~~~~~~~~~~~~~~~~~~~~~~~~~~~~~~~~~~~~~~~~~~~~
    \begin{titlepage}
        \title{Venique的校对说明\footnote{Made By \LaTeX}}
        \author{Venique}


        \maketitle
        \thispagestyle{empty}
        
        \vspace{0.55\textheight}
        
        \rightline{Venique}
        \rightline{venique@live.com}
    \end{titlepage}

    \newpage

    \pagestyle{fancy}
    \chead{Stage 2: 校对说明}
    \lhead{Venique}
    \rfoot{\thepage} 
    \cfoot{}
    \section{前请提要}
    	\subsection{目前进度梳理:}
    		在诸位的辛勤努力之下,我们终于完成了Stage 1的汉化工作。目前除了极少数没有完全翻译的文件之外,剩下所有的文件都已经给出了一个及以上的翻译。

    		在第一阶段翻译工作中,我们主要看重的是效率,因此有许多翻译存在着翻译腔、错字、语句不通顺的问题,在Stage 2阶段我们的目标是在原有翻译的基础上润色和校对

    \section{汉化总体目标}
    	这里主要写一下我希望达到的目标:
    	
    	时间表:

    	\begin{enumerate}[(a)]
    		\item 8.15 完成校审
    		\item 8.17 完成中文替换英文
    		\item 8.20 发送给Matt
    		\item 剩下的听天由命
    	\end{enumerate}


   	\section{下一步安排}
    	\subsection{Stage 2:校审格式}
    		\begin{enumerate}[情况1.]
    			\item \label{situa:1}某文件开始校审,且\textbf{校审完毕}。

    				按照如下格式填写文件头

    				\begin{center}
    					$\backslash$*Venique 校审, 基于 A,B,C, done*$\backslash$
    				\end{center}

    				其中A,B,C是本文件翻译人员

    				同时删除掉他们的翻译,仅保留自己的校审。因此校审过的文件会只剩下一行校审过的翻译。
    				(P.S Done是为了方便我写的统计程序识别)

    			\item 某文件开始校审,同时\textbf{未}校审完。

    				按照如下格式填写文件头

    				\begin{center}
    					$\backslash$*Venique 校审, 基于 A,B,C*$\backslash$
    				\end{center}

    				ABC说明同\ref{situa:1}


    				\textbf{请务必在校审完此文件后修改文件头},形式同\ref{situa:1}

    		\end{enumerate}

    	\subsection{Stage 2:工作量分配}
    		\begin{enumerate}
    			\item 目前总共有252个文件,部分文件为配置文件,按照时间表大概有18天的时间,因此每天至少要有14个文件校对完毕。
    		\end{enumerate}

    	\subsection{Stage 2: 校对人员}
			\begin{enumerate}
    			\item Stage 2 采取报名参与的形式,想参加Stage 2的人请在QQ上找我报名获取新的git仓库地址与用户名密码。
    			\item 在报名的时候需要提供一下信息
    				\begin{enumerate}
		    			\item 自己的在Stage 1中Git用的用户名
		    			\item 每天\textbf{平均}翻译量(要求每3天一平均都能达到这个翻译量,我会写一个脚本进行检测)
		    		\end{enumerate}
		    		在我审核通过后,会通过QQ和Email的方式发送Git仓库地址与用户名密码。
    		\end{enumerate}    		

    \section{其他说明:}
    	\begin{enumerate}
			\item 在校对的时候遇到翻译与自己的理解不同,请与原翻译讨论,请不要强行修改对方的翻译。校对仅仅是润色。。并不是可以随便改变原翻译的意思。

			\item 我好饿,求拍打喂食。

			\item 我戒了屁股!!难以置信。。。

			\item 目前的Git仓库,我会择日修改密码,停止工作。最近请注意及时上传自己的修改

		\end{enumerate}

\end{document}

